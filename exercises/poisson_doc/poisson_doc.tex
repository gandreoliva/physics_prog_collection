% Documentation for the Poisson equation
% --------------------------------------
% pdflatex poisson_doc.tex

\documentclass{article}
\usepackage{amsmath}
\usepackage{geometry}


\let\toparrow\vec
\renewcommand{\vec}[1]{\toparrow{\mathbf{#1}}}
\newcommand{\uv}[1]{\hat{\mathbf{#1}}}



\begin{document}
{\huge Poisson equation}

\section{Incompressible inviscid flow}

Conservation of mass implies
\[ \frac{\partial\rho}{\partial t} = \vec \nabla \cdot (\rho \vec v)   \]
\[ \implies 0 = \rho \vec \nabla \cdot \vec v \implies \vec \nabla \cdot \vec v = 0 \]

We define a potential function $f$, such that $\vec v = \vec \nabla f$. The
conservation of mass is then
\[ \nabla^2 f = 0 \]

\section{Finite differences}

Using finite differences,
\[ \frac{f(x+dx,y)-2f(x,y)+f(x-dx,y)}{dx^2} + \frac{f(x,y+dy)-2f(x,y)+f(x,y-dy)}{dy^2} = 0 \]

we set $dx=dy$. Then, using indices,
\[ +4f(i,j) -f(i,j+1) - f(i,j-1) - f(i+1,j) - f(i-1,j) = 0 \]

This can be represented in \emph{stencil notation} $(i\rightarrow,j\downarrow)$ in the computational grid as

\begin{tabular}{c|c|c|c|c}
& & & & \\ \hline
& & -1 & & \\ \hline
& -1 & +4 & -1 & \\ \hline
& & -1 & & \\ \hline
& & & &
\end{tabular}

As an example, for a $3\times 3$ grid we have for the cell at the center

\begin{tabular}{|c|c|c|}
\hline
& -1 & \\ \hline
-1 & 4 & -1\\ \hline
& -1 & \\
\hline
\end{tabular}

If we deconstruct this grid, putting rows one after another,
we can write the Poisson equation applied at the cell at the center
\[ ( 0, -1, 0; -1, 4, -1; 0, -1, 0 )
\left(
\begin{matrix}
f_{11}\\
f_{12}\\
f_{13}\\
f_{21}\\
f_{22}\\
f_{23}\\
f_{31}\\
f_{32}\\
f_{33}
\end{matrix}
\right)
=
0
 \]

So, we get a ``dot product'' notation.

The cells of the grid represent only the inner points of the whole computational domain.
That is why we enumerated them as $\{1,2,3\}$. The cells with indices 0 and 4 are
the boundaries. In order to use the same ``dot product'' notation as above, we can just put the
boundary values at the right hand side of the equation. So, for $i=1,j=2$, we would have

\[ 4f(1,2) - f(1,3) - f(1,1) - f(2,2) - \underline{f(0,2)} = 0 \]

\[ ( -1, 4, -1; 0, -1, 0; 0, 0, 0 )
\left(
\begin{matrix}
f_{11}\\
f_{12}\\
f_{13}\\
f_{21}\\
f_{22}\\
f_{23}\\
f_{31}\\
f_{32}\\
f_{33}
\end{matrix}
\right)
=
f_{02}
 \]

So, this means that we can continue with the same pattern that we used for the
cell at the center, and we put the boundary cells in the right-hand side. When
we have all equations for all cells, we can solve the system of equations for
all the unknowns. We can write the system in matrix notation.

\section{Matrix notation}

Now, we stack all the equations together in a matrix. For our $3\times 3$ grid,

\begin{equation}
\left[
\begin{array}{c c c | c c c | c c c}
 4&-1& 0&-1& 0& 0& 0& 0& 0 \\
-1& 4&-1& 0&-1& 0& 0& 0& 0 \\
 0&-1& 4& 0& 0&-1& 0& 0& 0 \\ \hline
-1& 0& 0& 4&-1& 0&-1& 0& 0 \\
 0&-1& 0&-1& 4&-1& 0&-1& 0 \\
 0& 0&-1& 0&-1& 4& 0& 0&-1\\ \hline
 0& 0& 0&-1& 0& 0& 4&-1& 0 \\
 0& 0& 0& 0&-1& 0&-1& 4&-1\\
 0& 0& 0& 0& 0&-1& 0&-1& 4
\end{array}
\right]
\left(
\begin{matrix}
f_{11}\\
f_{12}\\
f_{13}\\
f_{21}\\
f_{22}\\
f_{23}\\
f_{31}\\
f_{32}\\
f_{33}
\end{matrix}
\right)
=
\left(
\begin{matrix}
f_{01}\\
f_{02}\\
f_{03}\\
0\\
0\\
0\\
f_{41}\\
f_{42}\\
f_{43}
\end{matrix}
\right)
+
\left(
\begin{matrix}
f_{10}\\
0\\
f_{14}\\
f_{20}\\
0\\
f_{24}\\
f_{30}\\
0\\
f_{34}
\end{matrix}
\right)
\label{matrixeq}
\end{equation}

The matrix (that we will call $A$) is a block matrix of the following form
(we show an example for a $4\times 4$ grid)
\[
A =
\left(
\begin{matrix}
B & -I & & \\
-I & B & -I & \\
 & -I & B & -I \\
 & & -I & B
\end{matrix}
\right)
\]

where $I$ is the identiy matrix and
\[
B =
\left(
\begin{matrix}
4 & -1 & & \\
-1 & 4 & -1 & \\
 & -1 & 4 & -1 \\
 & & -1 & 4
\end{matrix}
\right)
\]

The equation \ref{matrixeq} is in the form
\[ A \vec f = \vec d + \vec e \]
The vector $\vec f$ contains the unknowns of the system, and vector
$\vec c = \vec d + \vec e$ contains the known boundary conditions.

For a given cell $f_{ij}$, we shall think of the row that has the positive value
($+4$) as the coefficient of $f_{ij}$ as the row that is supposed to calculate the value of $f_{ij}$, and
we can think of the negative coefficients as the values of the neighbouring cells
that are necessary to determine the value of $f_{ij}$.

\section{Indices}
In order to make calculations, we need to use an index $l$ to consecutively
enumerate the entries on $\vec f$. For a $3\times 3$ grid,

\begin{tabular}{c c | c}
$i$ & $j$ & $l$ \\ \hline
1 & 1 & 1 \\
1 & 2 & 2 \\
1 & 3 & 3 \\
2 & 1 & 4 \\
2 & 2 & 5 \\
2 & 3 & 6 \\
3 & 1 & 7 \\
3 & 2 & 8 \\
3 & 3 & 9
\end{tabular}

The definition of $l$ can be inferred by induction:
\[ l = (i-1)n + j \]

By the definition of modulo ($\%$) and integer division ($//$),
\[ i-1 = l//n \]
\[ j = l \% n \]
but, for the case of an exact division (modulo 0), we need to readjust the
indices, since there is no 0 index in $j$ for our numbering system of inner cells:
\begin{verbatim}
if (j == 0) then
  j = n
  i = i-1
end if
\end{verbatim}

The boundary conditions $f(0,j)$ of $\vec d$ can be set by setting $i=1$ and iterating
over $j$; the value is assigned to $d_l$. We proceed in a similar way for the other
boundary conditions: $f(n+1,j)\to d(n,j)$, $f(i,0)\to d(i,1)$, $f(i,n+1) \to d(i,n)$.

Now, in order to build the block matrix, we need two indices, $l$ and $m$, and we
need to determine the range of cells of $A$ that should contain the matrices $B$ and
$-I$.

Let's focus only on the blocks on the index $l$ (horizontal). The first block goes from $l=1$
to $l=n$. The second block starts from $l=n+1$ and ends at $l=2n$. The third block
goes from $l=2n+1$ and ends at $l=3n$. So, for each block $i=1,2,3,...$, the
corresponding $l$ for the start point is determined by $l = (i-1)n+1 $, and
the $l$ for the end point is $l = in$. The same must be done with the indices
$m$, as a function of $j$.

To reiterate: the matrix $A$ has $n^2$ rows (index $l$). Each row is composed of the $n^2$
cells of the whole computational grid, organized one row (of the grid) after another,
and described by the index $l$. The row $l$ is used to determine the value of the
cell $m=l$ (the diagonals represent the unknowns). Each linear equation (row $l$)
makes reference to neighbouring cells, for which the columns $m\neq l$ are used.


\section{Uniform free flow}

For a uniform free flow from right to left,
\[ \vec v = u \uv x  = - \vec \nabla f = \mathbf{const} \]
\[ u \uv x = \frac{\partial f}{\partial x}\uv x + \frac{\partial f}{\partial y}\uv y \]
\[ \implies u = \frac{\partial f}{\partial x} \]
\[ \implies f(x)  = ux + k_0 \]
Let's select $k_0$ (integration constant) as zero. Then, at the left boundary
$f(i,0) = 0$. At the right boundary, $f(i,n) = un = \text{const}$.
At the top boundary, $f(0,j) = ux$, and at the bottom boundary, $f(n,j) = ux$.



\section{Obstacle in a fluid flow}

Let us consider an obstacle in the flow. At the boundaries of the obstacle,
(inner boundary of the computational domain), the velocity should
``slide along the surface'', that is, $\vec v$ should be perpendicular to a vector $\uv n$ which
is perpendicular to the surface in the direction of ``outside'' the obstacle:
\begin{equation} \vec v \cdot \uv n = 0 \label{innerboundary} \end{equation}

Consider an obstacle that spans over only one grid cell at $(k_a,k_b)$.

\begin{tabular}{c|c|c|c|c}
& & & & \\ \hline
& & $k_a-1,k_b$ & & \\ \hline
& $k_a,k_b-1$ & $k_a,k_b$ & $k_a,k_b+1$ & \\ \hline
& & $k_a+1,k_b$ & & \\ \hline
& & & &
\end{tabular}

We need to delete the equation (row) on matrix $A$ that would determine the value for
$(k_a,k_b)$. In order not to create a singular matrix, we can simply put
\[ f_{k_a,k_b} = 0 \]

Now, we need to take care of the boundaries of the obstacle, that must obey
equation \ref{innerboundary}. First, we start with the ``cross'' (+) boundary cells that
we represented in the grid above. We need to remove from matrix $A$ the rows that
determine the values of each of the cells of the cross.

Now, we replace the row with the boundary condition. For example, let's take the
cell $(k_a,k_b-1)$. According to equation \ref{innerboundary}, it should obey
the condition
\[ (v_x \uv x + v_y \uv y) \cdot (-\uv x) = 0 \]
\[ \implies -v_x = 0 \]
\[ \implies \frac{f(k_a,k_b-2)-f(k_a,k_b-1)}{dx} = 0\]
so we end up with the simpler condition
\[ f(k_a,k_b-2) = f(k_a,k_b-1)  \]
which in general, means that the value of $f$ near the inner boundary should be just
copied from the neighbouring outer cell. Rearranging,
\[ -f(k_a,k_b-2) + f(k_a,k_b-1) = 0 \]

We can implement this in the code in the following way:
\begin{itemize}
\item The matrix $A$ has entries $A(l,m) = A((i_1,j_1)\to l,\ (i_2,j_2)\to m)$.
\item The row that determines $f_{k_a,k_b-1}$ is $A((k_a,k_b-1)\to l,\ :)$ and should be
crossed out. We need to replace it.
\item The entry in row $l$ that represents $f_{k_a,k_b-1}$ is $A(l,l)$, and its
value is now $1$.
\item The entry in row $l$ that represents $f_{k_a,k_b-2}$ is
$A(l,\ (k_a,k_b-2)\to m)$, and its value is now $-1$.
\end{itemize}

We can repeat the algorithm to match the other cross boundaries of the obstacle.

\vspace{2em}

For the corners, the situation is more complicated.

\begin{tabular}{c|c|c|c|c}
& & & & \\ \hline
& $k_a-1,k_b-1$ & & $k_a-1,k_b+1$ & \\ \hline
& & $k_a,k_b$ & & \\ \hline
& $k_a+1,k_b-1$ & & $k_a+1,k_b+1$ &\\ \hline
& & & &
\end{tabular}

In general, we should use the equation \ref{innerboundary}. Suppose we have surface
that makes an angle $\phi$ with the vertical. Then, it's easy to show that
\[ \uv n = \cos\phi \uv x + \sin\phi \uv y \]
and then we could solve the dot product
\[ \left(\frac{\partial f}{\partial x} \uv x + \frac{\partial f}{\partial y}\uv y\right)
\cdot
(\cos\phi \uv x + \sin\phi \uv y)
=0 \]
and create a condition that depends on the derivatives along the $i$ and $j$ directions,
For example, for $\phi = 45^\circ$ and the bottom right corner,
\[ \frac{1}{\sqrt{2}}\frac{f(i+2,j)-f(i+1,j)}{dx} +
\frac{1}{\sqrt{2}}\frac{f(i,j+2)-f(i,j+1)}{dy} = 0\]
This condition involves the neighbouring cells in the $i$ and $j$ directions, but
not in the diagonal direction.

For our very simple case of having only one cell as obstacle, it is simpler to
instead take directly the ``radial'' direction,
that is, using the diagonal direction. For example, for the bottom right corner,
\[ \frac{\partial f}{\partial r} = 0 \implies \]
\[ f(k_a+1,k_b+1) = f(k_a+2,k_b+2) \]

After replacing the corresponding rows in $A$, the system of equations is complete
and the solution can be calculated.


\section{Obstacle files}
A one cell obstacle can be described by the following grid:

\begin{tabular}{|c||c|c|c|c|c|}
\hline
	&-2	&-1	&0	&1	&2	\\ \hline\hline
2	&C	&	&c	&	&C	\\ \hline
1	&	&x	&+	&x	&	\\ \hline
0	&c	&+	&O	&+	&c	\\ \hline
-1	&	&x	&+	&x	&	\\ \hline
-2	&C	&	&c	&	&C	\\ \hline
\end{tabular}

where O is the obstacle cell, + are the ``cross'' boundary cells, that copy the
horizontal/vertical neighbors (c), and x are the cells that copy the diagonal
neighbors (C).

If we specify the center of the obstacle as having the coordinates $k_a,k_b$,
the specification for the operations necessary to handle the inner boundary can
be read from a file organized by columns:

\begin{enumerate}
	\item $\Delta k_a$ for the value of $l$
	\item $\Delta k_b$ for the value of $l$
	\item $\Delta k_a$ for the value of $m$
	\item $\Delta k_b$ for the value of $m$
	\item value in the system of equations.
\end{enumerate}

This is used in the code as follows:

The obstacle cell (coordinates \verb|0 0|, i.e., $k_a+0,k_b+0$) has the
potential $f$ set to zero, so, $A((k_a,k_b)\to l, (k_a,k_b)\to m) = 1$ (n.b.,
$m=l$). This is equivalent to the equation $f(k_a,k_b) = 0$. This means that the
line first line of the file is
\begin{verbatim}
0	0	0	0	1
\end{verbatim}

For an inner boundary cell, for example, \verb|0 -1| (which means
$k_a+0,k_b-1$), the equation to be satisfied is
\[ -f(k_a,k_b-2) + f(k_a,k_b-1) = 0 \]
So, $A((k_a,k_b-1)\to l, (k_a,k_b-1)\to m) = 1$, which means, in the file:
\begin{verbatim}
0	-1	0	-1	1
\end{verbatim}
and for the neighbouring cell,
$A((k_a,k_b-1)\to l, (k_a,k_b-2)\to m) = -1$, which is set in the file as
\begin{verbatim}
0	-1	0	-2	-1
\end{verbatim}

The same procedure has to be done for the other cells that have either a + or an
x in the diagram.

\emph{Warning:} when the matrix $A$ is built, no obstacle exists yet. When the
obstacle is added, the row $l$ of $A$ that contains every cell of the boundary
has to be set to zero before the new equations can be set. In the current
implementation of the code, that operation is done when $l = m$, that is, in the
previous example, the line \verb|0	-1	0	-1	1| has to come before the line
\verb|0	-1	0	-2	-1|.


\end{document}
